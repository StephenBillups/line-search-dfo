\documentclass{article}

\usepackage{amsmath}
\usepackage{amsfonts}
\usepackage{amssymb}
\usepackage[utf8]{inputenc}
\usepackage[english]{babel}
\usepackage{amsthm}
\usepackage[margin=0.5in]{geometry}

\newtheorem{theorem}{Theorem}

\title{Proof}
\author{Trever Hallock}

\newcommand{\grad}{\nabla f}
\newcommand{\xk}{{x}^{(k)}}
\newcommand{\ints}{\mathbb N}
\newcommand{\dk}{\Delta_k}
\newcommand{\rk}{\rho_k}
\newcommand{\pik}{\chi_k}
\newcommand{\sk}{{s^{(k)}}}
\newcommand{\mk}{{m_f^{k}}}
\newcommand{\xkpo}{{{x}^{(k+1)}}}
\newcommand{\dkpo}{\Delta_{k+1}}
\newcommand{\gk}{{g^{(k)}}}
\newcommand{\tone}{\omega_{\text{dec}}}
\newcommand{\ttwo}{\omega_{\text{inc}}}
\newcommand{\oalpha}{\tau_{\Delta}}


\begin{document}

\maketitle

\section{Hypothesis}
\subsection{H1}
The function $f$ is differentiable and its gradient $\grad$ is Lipschitz continuous with constant $L > 0$ in $ \Omega $.
\subsection{H2}
The function $f$ is bounded below in $ \Omega $.
\subsection{H3}
The matrices $H_k$ are uniformly bounded, that is, there exists a constant $ \beta \ge 1 $ such that $\|H_k\| \le \beta - 1$ for all $k \ge 0$.
\subsection{H4}
There exists a constant $c_2 > 0$ such that $ \| \gk - \grad(\xk) \| \le c_2 \dk $ for $k \in \ints $.

\section{Requirements}

\begin{align*}
\mk(\xk) - \mk(\xk + \sk) \ge c_1 \pik \min\{\frac{\pik}{1 + \|H_k\|}, \dk, 1\}
\end{align*}

\section{Constants}

\begin{align*}
x^{(0)} \in \Omega \\
\oalpha > 0 \\
\Delta_0 > 0 \\
0 < \tone < 1 \le \ttwo \\
\eta_1 \in (0, 1) \\
0 \le \eta < \eta_1 \le \eta_2
\end{align*}

\section{Definitions}

\begin{align*}
\rk = \frac{f(\xk) - f(\xk + \sk)}{\mk(\xk) - \mk(\xk + \sk)} \\
\mk(\xk + \sk) = f(\xk) + (\gk)^T \sk + \frac 1 2 \sk^T H_k \sk \\
\pik = \|P_{\Omega}(\xk - \gk) - \xk \| \\
S = \{k \in \ints | \rk > \eta \} \\
\bar{S} = \{k \in \ints | \rk \ge \eta_1 \} \\
c = \frac{L + c_2 + \frac {\beta} 2}{c_1} \\
c_0 = L + c_2 + \frac {\beta} 2 \\
\mathcal K = \big \{ k \in \ints | \dk \le \min \{ \frac {\pik}{\beta}, \frac{1-\eta_1}{\pik}c, \oalpha \pik, 1 \} \big \}
\end{align*}

\section{Copy}

\subsection{Lemma 3.1}
\begin{theorem}
Suppose that Hypothesis H1, H2, H4. If $k \in \mathcal K$, then $k \in \bar{S}$.
\end{theorem}
 
\begin{proof}

By the Mean Value Theorem, there exists a $t_k \in (0, 1)$ such that
\begin{align*}
f(\xk + \sk) = f(\xk) + \grad(\xk + t_k\sk)^T\sk
\end{align*}

By H1, H3, H4,
\begin{align*}
|f(\xk) - f(\xk + \sk) - (\mk(\xk) - \mk(\xk + \sk)| \\
= |-(\grad(\xk + t_k\sk) - \gk)^T\sk + \frac 1 2 (\sk)^T H_k \sk| \\
\le (\| \grad(\xk + t_k\sk) - \grad(\xk) \| + \| \grad(\xk)-\gk \|) \|\sk\| + \frac 1 2 \|\sk\|^2\|H_k\| \\
\le (t_k L \|\sk\| + c_2\dk) \|\sk\| + \frac 1 2 \beta \|\sk\|^2
\end{align*}

Since $\| \sk \| \le \dk$ and $t_k \in (0, 1)$ we have that
\begin{align}
|f(\xk) - f(\xk + \sk) - (\mk(\xk) + \mk(\xk + \sk)| \le c_0 \dk^2
\end{align}

By the definition of $\mathcal K$, for every $k \in \mathcal K$ we have that $\dk \le \oalpha \pik$ and consequently $\pik > 0$.
By the efficiency condition, this means that $\mk(\xk) - \mk(\xk + \sk) \ne 0$.
Then,
\begin{align*}
|\rk - 1| = \bigg |\frac{f(\xk) - f(\xk + \sk) - (\mk(\xk) - \mk(\xk + \sk)}{\mk(\xk) - \mk(\xk + \sk)} \bigg | \\
\le \frac {c_0 \dk^2} {c_1 \pik \min\{\frac{\pik}{\beta}, \dk, 1\}} \\
= \frac {c \dk^2} {\pik \min\{\frac{\pik}{\beta}, \dk, 1\}}
\end{align*}

We also know that 
\begin{align*}
\dk = \min\{\frac {\pik} {\beta}, \dk, 1 \} \\
\frac {c \dk}{\pik} \le 1 - \eta_1
\end{align*}
so that
\begin{align*}
|\rk - 1| \le 1 - \eta_1 \\
\Longrightarrow \rk \ge \eta_1
\end{align*}
so that $k \in \bar{S}$.


\end{proof}



\subsection{Lemma 3.2}
\begin{theorem}
Suppose Hypothesis H2, H3. Then the sequence $(\dk)$ converges to zero.
\end{theorem}
 
\begin{proof}

Suppose that $\bar{S}$ is finite. Then there exists $k_0 \in \ints$ such that for all $k \ge  k_0$, $\dkpo \le \tone \dk$.
Thus, $(\dk)$ converges to zero.
From now on $\bar{S}$ is infinite.
For any $k \in \bar{S}$, we know $\dk \le \oalpha \pik$ using $H3$ we have
\begin{align*}
f(\xk) -  f(\xkpo) \ge \eta_1 \big (\mk(\xk) - \mk(\xk + \sk)\big ) \ge \eta_1 c_1 \pik \min\{\frac{\pik}{\beta}, \dk, 1\}\\
f(\xk) - f(\xkpo) \ge \eta_1c_1\frac{\dk}{\oalpha}\min\{\frac{\dk}{\oalpha \beta}, \dk, 1\}
\end{align*}
Because $f(\xk)$ is nonincreasing, the left hand side goes to zero.
Thus,
\begin{align}
\lim_{k \in \bar{S}} \dk = 0
\end{align}


Consider the set
$\mathcal U = \{ k \in \ints | k \not \in \bar S \}$.
If $\mathcal U$ is finite, then $\lim_{k\to\infty}\dk = 0$.
Otherwise, consider $k \in \mathcal U$ and define $\l_k$ to be the last index in $\bar S$ before $k$.
Then $l_k$ is well-defined for all large $k$  and $\dk \le \ttwo \Delta_{l_k}$ which implies that
\begin{align}
\lim_{k \in \mathcal U } \dk \le \ttwo \lim_{k \in \mathcal U} \Delta_{l_k} = \ttwo \lim_{l_k \in \bar{S}} \Delta_{l_k}
\end{align}

so that $\lim_{k \in \mathcal U} \dk = 0$.

\end{proof}




\subsection{Lemma 3.3}
\begin{theorem}
Suppose that Hypothesis H1, H4. Then $\liminf_{k\to\infty} \pik = 0$.
\end{theorem}
 
\begin{proof}
Suppose for a contradiction that there exists a constants $\epsilon > 0$ and an integer $K > 0$ such that for $\pik \ge \epsilon$ for each $k \ge K$.
Take $ \tilde \Delta = \min \{\frac{\epsilon}{\beta}, \frac{(1 - \eta_1)c}{\epsilon}, \oalpha \epsilon, 1\}$.
Consider $k \ge K$.
If $\dk \le \tilde \Delta$, then $k \in \mathcal K$.
Then $k \in \bar S$ and thus $\dkpo \ge \dk$.
Therefore, the trust region radius can only decrease if $\Delta > \tilde \Delta$, and in this case $\dkpo = \tone\dk > \tone \tilde \Delta$.
Therefore, one can see that for all $k \ge K$
\begin{align}
\dk \ge \min\{\tone \tilde \Delta, \dk \}
\end{align}
which is a contradiction.
\end{proof}



\subsection{Lemma 3.4}
\begin{theorem}
Suppose that H1 to H4 and $\eta > 0$. Then $\lim_{k\to\infty}\pik=0$.
\end{theorem}

\begin{proof}
Suppose for a contradiction that for some $\epsilon > 0$ the set $\ints ' = \{k \in \ints | \pik \ge \epsilon \}$
is finite.

Because $\lim_{k\to\infty}\Delta_k\to 0$, there exists a $k_0 \in \ints$ such that for all $k \ge k_0$,

\begin{align*}
\dk \le \min\{\frac{\epsilon}{\beta}, \frac{(1-\eta_1)\epsilon}{c}, \oalpha\pik, 1\}.
\end{align*}

Then if $k \in \ints '$ with $k \ge k_0$:

\begin{align*}
\dk \le \min\{\frac{\pik}{\beta}, \frac{(1-\eta_1)\pik}{c}, \oalpha\pik, 1\}
\end{align*}

and therefore $k \in \bar S \subset S$.

Given $k \in \ints'$ with $k\ge k_0$, consider $l_k$ the first index such that $l_k > k$ and $\chi_{l_k} \le \frac{\epsilon} 2$.
The existence of $l_k$ is ensured by lemma 3.3.
This means that $\pik - \chi_{l_k} \ge \frac {\epsilon} 2 $.
Using the definition of $\pik$, the triangle inequality and the contraction property of projections, we have that

\begin{align*}
\frac{\epsilon}{2} \le \|P_{\Omega}(\xk - \gk) - \xk\| - \|P_{\Omega}(x^{(l_k)} - g^{(l_k)}) - x^{(l_k)}\| \\
\le \|P_{\Omega}(\xk - \gk) - \xk - P_{\Omega}(x^{(l_k)} - g^{(l_k)}) + x^{(l_k)}\| \\
\le 2\|\xk - x^{(l_k)}\| + \|\gk - g^{(l_k)}\| \\
=   2\|\xk - x^{(l_k)}\| + \|\gk - \grad(\xk) + \grad(\xk) - \grad(x^{(l_k)}) + \grad(x^{(l_k)}) - g^{(l_k)}\| \\
\le 2\|\xk - x^{(l_k)}\| + \|\gk - \grad(\xk)\| + \|\grad(\xk) - \grad(x^{(l_k)})\| + \|\grad(x^{(l_k)}) - g^{(l_k)}\|.
\end{align*}

So that
\begin{align}
\frac{\epsilon} 2 \le (2 + L) \|\xk - x^{(l_k)}\| + c_2(\dk + \Delta_{l_k}).
\end{align}

Consider $C_k = \{i \in S | k \le i < l_k\}$.
Note that because $k \in S$, so $C_k \ne \varnothing $.
For each $i \in C_k$, using the fact that $i \in S$, and H3, we conclude that 

\begin{align}
f(x^{(i)}) - f(x^{(i+1)}) \ge \eta\big ( \mk(x^{(i)}) - \mk(x^{(i)} + s^{(i)}) \big ) \ge \eta c_1 \chi_i \min\{\frac{\chi_{i}}{\beta}, \Delta_i, 1\} 
\end{align}

By the definition of $l_k$, we have that $\chi_i > \frac{\epsilon}{2}$ for all $i \in C_k$.
As $i \ge k$, $\Delta_i \le \frac{\epsilon}{\beta}$ and $\Delta_i \le 1$.
Therefore,
\begin{align}
\frac{\Delta_i}{2} \le \frac{\epsilon}{2\beta} \le \frac{\pi_i}{\beta}.
\end{align}

It follows that
\begin{align}
f(x^{(i)}) - f(x^{(i+1)}) > \frac{\eta c_1 \epsilon \Delta_i}{4}
\end{align}

and hence
\begin{align}
\Delta_i < \frac{4}{\eta c_1 \epsilon} \big ( f(x^{(i)}) - f(x^{(i+1)})\big ).
\end{align}

Meanwhile,
\begin{align}
\|x^{(i)} - x^{(l_k)}\| \le \sum_{i \in C_k}\|x^{(i)} - x^{(i+1)}\| \le \sum_{i \in C_k} \Delta_i
\end{align}

so that

\begin{align}
\|x^{(i)} - x^{(l_k)}\| < \frac{4}{\eta c_1 \epsilon} \big ( f(x^{(i)}) - f(x^{(i+1)})\big ).
\end{align}

We also know that $(f(\xk))$ is bounded below, and since it is nonincreasing, $f(\xk)  - f(x^{(l_k)}) \to 0$.
Therefore $(\|\xk - x^{(l_k)}\|)_{k \in \ints '}$ converges to zero.

\end{proof}


\begin{theorem}
Suppose that H1 to H4 hold.

If $\eta = 0$, then
\begin{align}
\liminf_{k\to\infty} \|P_{\Omega}(\xk - \grad(\xk)) - \xk \| = 0.
\end{align}

If $\eta > 0$, then
\begin{align}
\lim_{k\to\infty} \|P_{\Omega}(\xk - \grad(\xk)) - \xk \| = 0.
\end{align}

\end{theorem}

\begin{proof}
By the triangle inequality, the contraction property of projections and H4, we have that

\begin{align*}
\|P_{\Omega}(\xk - \grad(\xk)) - \xk \| = \|P_{\Omega}(\xk - \grad(\xk)) - P_{\Omega}(\xk - \gk) + P_{\Omega}(\xk - \gk) - \xk\| \\
\le \|P_{\Omega}(\xk - \grad(\xk)) - P_{\Omega}(\xk - \gk)\| + \|P_{\Omega}(\xk - \gk) - \xk\| \\
\le \|\grad(\xk) - \gk\| + \|P_{\Omega}(\xk - \gk) - \xk\| \\
\le c_2 \Delta_k + \pik
\end{align*}

\end{proof}




\end{document}








































