


\usepackage{amsmath}
\usepackage{graphicx}
\usepackage{color}
%\usepackage{algorithm}
%\usepackage[noend]{algpseudocode}
%\usepackage{varwidth}% http://ctan.org/pkg/varwidth
\usepackage{xspace}
\usepackage{cite}
%\usepackage{placeins}
\usepackage[margin=1in]{geometry}
\usepackage{amsfonts}
\usepackage{array,multirow}
\usepackage{amssymb,amsmath,amsthm}
\usepackage[]{algorithmicx}
\usepackage{algpseudocode} 
\usepackage{enumitem}
\usepackage[capitalise,nameinlink,noabbrev]{cleveref}
\usepackage[normalem]{ulem}	% For color comments
\usepackage{float}
\floatstyle{ruled}
\newfloat{algorithm}{thp}{lop}
\floatname{algorithm}{Algorithm}



%============================

\newtheorem{theorem}{Theorem}[section]
\newtheorem{corollary}{Corollary}[theorem]
\newtheorem{lemma}[theorem]{Lemma}
\newtheoremstyle{case}{}{}{}{}{}{:}{ }{}
\theoremstyle{case}
\newtheorem{case}{Case}


%=======================================
% Steve's definitions for marking up
%=======================================

\newcommand{\new}[1]{{\color{blue}#1}}
\newcommand\hcancel[2][black]{\setbox0=\hbox{$#2$}\rlap{\raisebox{.45\ht0}{\textcolor{#1} {\rule{\wd0}{1pt}}}}#2} 
\newcommand{\replace}[2]{{\color{red}\sout{#1}\color{black}{\color{red}#2\color{black}}}} %TeX source markup.
% \newcommand{\replace}[2]{{{\color{red}#2\color{black}}}} %TeX source markup.
\newcommand{\replaceb}[2]{{\color{blue}\sout{#1}\color{black}{\color{blue}#2\color{black}}}} %TeX source markup.
\newcommand{\replacemath}[2]{{\hcancel[red]{#1}{}{\color{red}#2\color{black}}}} %TeX source markup.
% \newcommand{\replacemath}[2]{{\color{red}#2\color{black}}} %TeX source markup.
\newcommand{\replacemathb}[2]{{\hcancel[blue]{#1}{}{\color{blue}#2\color{black}}}} %TeX source markup.
\newcommand{\sbnote}[1]{\textsf{{\color{cyan}{ SCB note:}   #1} }\marginpar{{\textbf{Comment}}}}

%================================
% Other macros added by Steve
%
\newcommand{\domain}{X}
\newcommand{\real}{\mathbb R}
\newcommand{\norm}[1]{\| #1 \|}
\newcommand{\union}{\cup}
\newcommand{\intersect}{\cap}


% Added for consistency

\newcommand{\modelk}{{{m}_f}^{(k)}}
\newcommand{\modelkmone}{{{m}_f}^{(k-1)}}
\newcommand{\modelconstrainti}{{{m}_{c_i}}^{(k)}}
\newcommand{\modelconstraint}{{{m}_{c}}^{(k)}}
\newcommand{\iteratek}{{x}^{(k)}}
\newcommand{\trialk}{{s}^{(k)}}
\newcommand{\iteratekpone}{{x}^{(k+1)}}
\newcommand{\innertrk}{{T^{(k)}}_{\text{in}}}
\newcommand{\outertrk}{{T^{(k)}}_{\text{out}}}
\newcommand{\feasible}{{F}}
\newcommand{\feasiblek}{{F}^{(k)}}
\newcommand{\ellipsek}{{E^{(k)}}}
\newcommand{\chik}{{\chi^{(k)}}}
\newcommand{\xik}{{\xi^{(k)}}}
\newcommand{\gradmodelk}{\nabla{{m}_f}^{(k)}}


\newcommand{\ptx}{p(t,\iteratek)}
\newcommand{\Px}{P_X}
\newcommand{\ptjxk}{p(t_j, \iteratek)}
\newcommand{\tj}{t_j}
\newcommand{\tgc}{{{t}^{(k)}}_{GC}}
\newcommand{\gck}{{{x}^{(k)}}_{GC}}
\newcommand{\sgck}{{{s}^{(k)}}_{GC}}
\newcommand{\xj}{{{x}^{(k)}}_{j}}
\newcommand{\sj}{{{s}^{(k)}}_{j}}


\newcommand{\innerfritr}{D_{\text{in}}}
\newcommand{\outerfritr}{D_{\text{out}}}
\newcommand{\omegainc}{\omega_{\text{inc}}}
\newcommand{\omegadec}{\omega_{\text{dec}}}
\newcommand{\gammasm}{\gamma_{\text{min}}}
\newcommand{\gammabi}{\gamma_{\text{sufficient}}}
\newcommand{\ximin}{\xi_{\text{min}}}


\DeclareMathOperator*{\argmin}{arg\,min}
\DeclareMathOperator*{\argmax}{arg\,max}
