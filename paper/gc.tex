\documentclass{article}



% \DeclareMathOperator*{\argmin}{arg\,min}

%\newtheorem{theorem}{Theorem}[section]
%\newtheorem{corollary}{Corollary}[theorem]
%\newtheorem{lemma}[theorem]{Lemma}
%\newtheoremstyle{case}{}{}{}{}{}{:}{ }{}
%\theoremstyle{case}
%\newtheorem{case}{Case}

%\newcommand{\new}[1]{{\color{blue}#1}}
%\newcommand\hcancel[2][black]{\setbox0=\hbox{$#2$}\rlap{\raisebox{.45\ht0}{\textcolor{#1} {\rule{\wd0}{1pt}}}}#2} 
%\newcommand{\replace}[2]{{\color{red}\sout{#1}\color{black}{\color{red}#2\color{black}}}} %TeX source markup.
% \newcommand{\replace}[2]{{{\color{red}#2\color{black}}}} %TeX source markup.
%\newcommand{\replaceb}[2]{{\color{blue}\sout{#1}\color{black}{\color{blue}#2\color{black}}}} %TeX source markup.
%\newcommand{\replacemath}[2]{{\hcancel[red]{#1}{}{\color{red}#2\color{black}}}} %TeX source markup.
% \newcommand{\replacemath}[2]{{\color{red}#2\color{black}}} %TeX source markup.
%\newcommand{\replacemathb}[2]{{\hcancel[blue]{#1}{}{\color{blue}#2\color{black}}}} %TeX source markup.
%\newcommand{\sbnote}[1]{\textsf{{\color{cyan}{ SCB note:}   #1} }\marginpar{{\textbf{Comment}}}}





\usepackage{amsmath}
\usepackage{graphicx}
\usepackage{color}
%\usepackage{algorithm}
%\usepackage[noend]{algpseudocode}
%\usepackage{varwidth}% http://ctan.org/pkg/varwidth
\usepackage{xspace}
\usepackage{cite}
%\usepackage{placeins}
\usepackage[margin=1in]{geometry}
\usepackage{amsfonts}
\usepackage{array,multirow}
\usepackage{amssymb,amsmath,amsthm}
\usepackage[]{algorithmicx}
\usepackage{algpseudocode} 
\usepackage{enumitem}
\usepackage{longtable}
\usepackage[capitalise,nameinlink,noabbrev]{cleveref}
\usepackage[normalem]{ulem}	% For color comments
\usepackage{float}
\usepackage{mathtools}
\floatstyle{ruled}
\newfloat{algorithm}{thp}{lop}
\floatname{algorithm}{Algorithm}



%============================

\newtheorem{theorem}{Theorem}[section]
\newtheorem{corollary}{Corollary}[theorem]
\newtheorem{lemma}[theorem]{Lemma}
\newtheoremstyle{case}{}{}{}{}{}{:}{ }{}
\theoremstyle{case}
\newtheorem{case}{Case}


%=======================================
% Steve's definitions for marking up
%=======================================

\newcommand{\new}[1]{{\color{blue}#1}}
\newcommand\hcancel[2][black]{\setbox0=\hbox{$#2$}\rlap{\raisebox{.45\ht0}{\textcolor{#1} {\rule{\wd0}{1pt}}}}#2} 
\newcommand{\replace}[2]{{\color{red}\sout{#1}\color{black}{\color{red}#2\color{black}}}} %TeX source markup.
% \newcommand{\replace}[2]{{{\color{red}#2\color{black}}}} %TeX source markup.
\newcommand{\replaceb}[2]{{\color{blue}\sout{#1}\color{black}{\color{blue}#2\color{black}}}} %TeX source markup.
\newcommand{\replacemath}[2]{{\hcancel[red]{#1}{}{\color{red}#2\color{black}}}} %TeX source markup.
% \newcommand{\replacemath}[2]{{\color{red}#2\color{black}}} %TeX source markup.
\newcommand{\replacemathb}[2]{{\hcancel[blue]{#1}{}{\color{blue}#2\color{black}}}} %TeX source markup.
\newcommand{\sbnote}[1]{\textsf{{\color{cyan}{ SCB note:}   #1} }\marginpar{{\textbf{Comment}}}}

%================================
% Other macros added by Steve
%
\newcommand{\domain}{X}
\newcommand{\real}{\mathbb R}
\newcommand{\norm}[1]{\| #1 \|}
\newcommand{\union}{\cup}
\newcommand{\intersect}{\cap}


% Added for consistency

\newcommand{\modelk}{{{m}_f}^{(k)}}
\newcommand{\modelkmone}{{{m}_f}^{(k-1)}}
\newcommand{\modelconstrainti}{{{m}_{c_i}}^{(k)}}
\newcommand{\modelconstraint}{{{m}_{c}}^{(k)}}
\newcommand{\iteratek}{{x}^{(k)}}
\newcommand{\trialk}{{{s}^{(k)}}}
\newcommand{\iteratekpone}{{x}^{(k+1)}}
\newcommand{\outertrk}{{T_{\text{out}}^{(k)}}}
\newcommand{\searchtrk}{{T_{\text{search}}^{(k)}}}
\newcommand{\sampletrk}{{T_{\text{interp}}^{(k)}}}
\newcommand{\innertrk}{\color{red} \text{Replace me} \color{black} }
\newcommand{\feasible}{{F}}
\newcommand{\feasiblek}{{F}^{(k)}}
\newcommand{\ellipsek}{{E^{(k)}}}
\newcommand{\chik}{{\chi^{(k)}}}
\newcommand{\xik}{{\xi^{(k)}}}
\newcommand{\gradmodelk}{\nabla{{m}_f}^{(k)}}


\newcommand{\ptx}{p(t,\iteratek)}
\newcommand{\Px}{P_X}
\newcommand{\ptjxk}{p(t_j, \iteratek)}
\newcommand{\tj}{t_j}
\newcommand{\tgc}{{{t}^{(k)}}_{GC}}
\newcommand{\gck}{{{x}^{(k)}}_{GC}}
\newcommand{\sgck}{{{s}^{(k)}}_{GC}}
\newcommand{\xj}{{{x}^{(k)}}_{j}}
\newcommand{\sj}{{{s}^{(k)}}_{j}}
\newcommand{\qk}{{Q^{(k)}}}


\newcommand{\innerfritr}{D_{\text{in}}}
\newcommand{\outerfritr}{D_{\text{out}}}
\newcommand{\omegainc}{\omega_{\text{inc}}}
\newcommand{\omegadec}{\omega_{\text{dec}}}
\newcommand{\gammasm}{\gamma_{\text{min}}}
\newcommand{\gammabi}{\gamma_{\text{sufficient}}}
\newcommand{\ximin}{\xi_{\text{min}}}
\newcommand{\polydn}{\mathcal{P}^d_n}

% Added for the proof of convergence.
% Should remove these.

\newcommand{\ints}{\mathbb N}
\newcommand{\dk}{\Delta_k}
\newcommand{\rk}{\rho_k}
\newcommand{\gk}{{\nabla m_f^{(k)}(x^{(k)})}}
\newcommand{\oalpha}{\tau_{\Delta}}
\newcommand{\hk}{{\nabla^2m_f^{(k)}(x^{(k)})}}

\newcommand{\grad}{\nabla f}
\newcommand{\xkpo}{{{x}^{(k+1)}}}
\newcommand{\dkpo}{\Delta_{k+1}}


\DeclareMathOperator*{\argmin}{arg\,min}
\DeclareMathOperator*{\argmax}{arg\,max}


% From find ellipse
\newcommand{\xk}{{x^{(k)}}}
\newcommand{\ak}{{A^{(k)}}}
\newcommand{\bk}{{b^{(k)}}}
\newcommand{\pik}{{\pi^{(k)}}}

\newcommand{\dbuf}{{\delta_{\text{buffer}}}}
\newcommand{\reg}{{\delta_{\text{regularity}}}}

\newcommand{\feasdir}{{u_{\text{feas}}}}
\newcommand{\hfeasdir}{{\hat{u}_{\text{feas}}}}


\newcommand{\xo}{{{\bar x}}}






\title{Generalized Cauchy Point Proof}
\author{Trever Hallock}

\begin{document}

\section{Notation}

%Assume that $\tj$ is a fixed constant defined later.
Let $\Px$ be the projection operator on to the convex set $X$.
Also, let $\mathcal T(x)$ and $\mathcal N(x)$ be the tangent cone and normal cone at $x$ with respect to $\feasiblek$ respectively.

The projected steepest descent path from iterate $\iteratek$ is defined as a function of $t$ by
\[ \ptx = P_{\feasiblek}[\iteratek-t\gradmodelk(\iteratek)] \]
which corresponds to a step
\[ \trialk(t) = p(t,\iteratek)-\iteratek. \]

We can minimize along this path 
\[\tgc = \argmin_{\{t\in\mathbb R|\|\trialk \| \le \Delta_k\}} \langle \gradmodelk(x), \ptx - \iteratek \rangle \]
to define the generalized cauchy point
\[\gck = p(\tgc, \iteratek)\]
\[\sgck = \gck-\iteratek.\]

We will also define a criticallity measure
\begin{align}
\label{chi_definition} \chik(x, \tau) = | \min_{\{d \in \mathbb R | x+d \in \feasiblek, \|d\| \le \tau\}}\langle \gradmodelk(x), d\rangle|
\end{align}
\[\chik(x) = \chik(x, 1)\]
\[\chik = \chik(\iteratek)\]
as well as
\[\beta_k = 1 + \max_{x\in\innertrk}\|\nabla^2\modelk(x)\|.\]

\section{Acceptable steps}
\label{close_enough}
In practice, we do not need to find the $\gck$, but only need to find some approximate solution $\tj$ that is close to the minimum.
To define this acceptable point $\xj = p(\tj, \iteratek)$ and corresponding step $\sj = \xj - \iteratek$, we first choose fixed constants
\begin{align}
0 < \kappa_{ubs} < \kappa_{lbs} < 1, \quad \kappa_{frd} \in (0, 1), \quad \text{and} \quad\kappa_{epp} \in (0, \frac 1 2 ).
\end{align}

We require that both
\begin{align}
\label{too_big_1}
\|\sj\|\le \Delta_k
\end{align}
\begin{center}and\end{center}
\begin{align}
\label{too_big_2}
\modelk(\xj) \le \modelk(\iteratek) + \kappa_{ubs}\langle \gradmodelk(\iteratek), \sj\rangle
\end{align}
are satisfied.
Additionally, we require that at least one of
\begin{align}
\label{too_small_1}
\|\sj\| \ge \kappa_{frd}\Delta_k
\end{align}
\begin{center}or\end{center}
\begin{align}
\label{too_small_2}
\modelk(\xj) \ge \modelk(\iteratek) + \kappa_{lbs}\langle \nabla\modelk(\iteratek), \sj\rangle
\end{align}
\begin{center}or\end{center}
\begin{align}
\label{too_small_3}
\|P_{\mathcal T(\xj)}[-\gradmodelk(\iteratek)]\| \le \kappa_{epp} \frac{\langle \nabla\modelk(\iteratek), \sj \rangle}{\Delta_k}
\end{align}
must be satisfied.



\section{Proofs}

\begin{theorem}
\label{12_1_3}
The function $\phi(t) = \|\ptx-x\|$ is nondecreasing for $t\ge0$.
If $\lim_{t\to\infty}\phi(t) < \infty$, then
\[
\lim_{t\to\infty}\|P_{\mathcal T(\ptx)}[-\nabla f(x)]\| = 0
\]
\end{theorem}

\begin{proof}
The first statement follows because the projection operator is monotone.

%For the second, choose an $\epsilon > 0$ such that $\epsilon \delta\|\gradmodelk(\iteratek)\| \le \frac 1 2 $
%where $\delta = \max_i\max$ and consider the points $x_i = p(i\epsilon, x)$.
%Define $\delta = \lim_{i\to\infty}\|x_i-x\|$.

\end{proof}


\begin{theorem}
\label{chi_non_inc}
\label{12_1_5_1}
The function $\chi(x,\theta)$ is nonincreasing as a function of $\theta$ for all $\theta>0$.
\end{theorem}

\begin{proof}
This follows immediately from the definition because $\chi$ is a minimum over a potentially larger domain, meaning that $d$ within the definition of $\chi$ in equation \ref{chi_definition} has more locations to minimize $\chi$.
\end{proof}

\begin{theorem}
\label{chi_non_inc}
The function $\frac{\chi(x,\theta)}{\theta}$ is nonincreasing as a function of $\theta$ for all $\theta>0$.
\end{theorem}

\begin{proof}
Consider two $0 < \theta_1 < \theta_2$  and two vectors $d_1$ and $d_2$ such that
$\xi(x, \theta_j) = -\langle\nabla \modelk(x), d_j\rangle$, $\|d_j\|\le\theta_j$, $x + d_j \in \feasiblek$, $j=1,2$.
Then $x + \frac {\theta_1}{\theta_2} d_2 \in [x, x + d_2]$ so that it is within the minimization problem for $\xi(x, \theta_1)$ and $d_1$.
Then

\[
\frac{\xi(x, \theta_1)}{\theta_1} \ge \frac 1 {\theta_1} |\langle\nabla\modelk(x), \frac{\theta_1}{\theta_2}\rangle| = \frac{\xi(x, \theta_2)}{\theta_2}
\]

\end{proof}

\begin{theorem}
We have that for any $d$ such that $x + d \in \feasiblek$
\begin{align}
\label{12_1_5_3}
\chik(x, \theta) \le |\langle\nabla \modelk, d\rangle| + 2\theta \|P_{T(x+d)}[-\nabla\modelk(x)]\|
\end{align}
for all $\theta > \|d\|$.
\end{theorem}

\begin{proof}

Let $d_{\star}$ be a solution of \ref{12_1_8}, then by definition \ref{chi_definition} and the triangle inequality, we have that
\begin{align}
\label{the_start}
\chi(x, \theta)\le \|\langle\nabla\modelk(x), d\rangle\| + \|\langle\nabla\modelk(x), d_{\star} - d\rangle\|.
\end{align}
Also,
\[
d_{\star}-d = (x+d_{\star}) - (x+d) \in \mathcal T(x+d)
\]
as $x+d_{\star} \in \feasiblek$ so that
\[
\langle P_{\mathcal N(x+d)}[-\nabla\modelk(x)], d_{\star} - d\rangle \le 0
\]


Then, because
\[
\nabla\modelk(x) = P_{\mathcal T(x+d)}[\nabla\modelk(x)] + P_{\mathcal N(x+d)}[\nabla\modelk(x)]
\]
we have
\[
\langle\nabla\modelk(x), d_{\star} - d\rangle = -\langle P_{\mathcal T(x+d)}[-\nabla\modelk(x)], d_{\star} - d\rangle - \langle P_{\mathcal N(x+d)}[-\nabla\modelk(x)], d_{\star} - d\rangle
\]
\[
\ge -\langle P_{\mathcal T(x+d)}[-\nabla\modelk(x)], d_{\star} - d\rangle.
\]

Thus
\[
|\langle\nabla\modelk(x), d_{\star} - d\rangle| \le \|\langle -P_{\mathcal T(x+d)}[-\nabla\modelk(x)], d_{\star} - d\rangle\| \le \|d_{\star} - d\|\|P_{\mathcal T(x+d)}[-\nabla\modelk(x)]\|.
\]

Now take the fact that
\[
d_{\star} - d =\le \|d_{\star}\|-\|d\| \le 2\theta
\]
and plug into \ref{the_start} to find the result.
\end{proof}


\begin{theorem}
\label{12_1_4}
For each point $\ptx$ on the projected-gradient path, $\ptx-x$ is a solution to the problem
\begin{align}
\label{12_1_8}
\min_{x+d\in \feasiblek, \|d\| \le \theta} \langle \nabla \modelk(x), d\rangle
\end{align}
where $\theta = \|\ptx - x\|$.
Furthermore, $\ptx - x$ is also a solution of this problem for all $\theta \ge \|\ptx - x\|$ whenever
\begin{align}
\label{12_1_9}
-\nabla \modelk(x) \in \mathcal N(\ptx)
\end{align}
\end{theorem}

\begin{proof}
Consider the problem of projecting $x-t\nabla \modelk(x)$ onto $\feasiblek$, that is

\begin{align}
\label{proj_to_feas}
\min_{x+u \in \feasiblek} \|x-t\nabla \modelk(x) - (x + u) \|^2 = \min_{x+u\in \feasiblek}\|t\nabla m_f(x) + u\|^2
\end{align}
If $t=0$, then $\ptx = x$ and the solution is $u=0$, so that $\theta = \ptx-x = 0$, $d=0$ is a solution.

Suppose that $t>0$ and that $u$ solves \ref{proj_to_feas}.
Note that the first order optimality conditions for this problem are
\[
-t\nabla \modelk(x)-u \in \mathcal N (x+u).
\]

Since $t>0$ and $\mathcal N(x+u)$ is a cone, this condition may be rewritten as 
\begin{align}
\label{rewritten}
-\frac 1 t u -\nabla \modelk(x) \in \mathcal N(x+u)
\end{align}
On the other hand, problem \ref{12_1_8} is equivalent to
\[\min_{x+d\in \feasiblek, \|d\|^2 \le \theta^2} \langle\nabla \modelk(x), d\rangle \]
whose first-order optimality conditions are given by some $\lambda \ge 0$ such that
\begin{align}
\label{12_1_11}
-2\lambda d-\nabla \modelk(x) \in N(x+d)
\end{align}
and
\begin{align}
\lambda(\|d\|^2-\theta^2) = 0.
\end{align}
The first of these conditions is identical to \ref{rewritten} if one sets $d=u$ and $\lambda=\frac 1 {2t}>0$.
The third immediately results from the identity 
\[
\theta=\|d\| = \|u\|=\|\ptx-x\|
\]
Now suppose that \ref{12_1_9} holds. Then $d=\ptx-x$ and $\lambda=0$ then satisfy \ref{12_1_11} for all $\theta \ge \|\ptx-x\|$, which concludes the proof.


\end{proof}





\begin{theorem}
Any point $ \xj $ that satisfies the criteria within section \ref{close_enough} also satisfies
\[
\modelk(\iteratek) - \modelk(\xj) \ge \kappa_{dcp} \chik \min\{\frac{\chik}{\beta_k}, \Delta_k, 1\}
\]
and
\begin{align}
\label{12_2_6}
\modelk(\iteratek) - \modelk(\xj) \ge \kappa_{ubs}|\langle \gradmodelk(\iteratek), \sj\rangle|
\end{align}

\end{theorem}

\begin{proof}

We see that \ref{12_2_6} follows directly from \ref{too_big_2}.
%\begin{align}
%\modelk(\gck) \le \modelk(\iteratek) + \kappa_{ubs}\langle \gradmodelk(\iteratek), \sgck \rangle \nonumber \\
%-\kappa_{ubs}\langle \nabla m_f(x^k), s_k(t_j)\rangle  \le m_k(x_k) - m_k(p(t_j, x_k))\nonumber \\
%\kappa_{ubs}|\langle \nabla m_f(x^k), s_k(t_j)\rangle|  \le |m_k(x_k) - m_k(p(t_j, x_k))| = m_k(x_k) - m_k(p(t_j, x_k))\nonumber 
%\end{align}

Also, Theorem \ref{12_1_4} gives us that
\begin{align}
|\langle \nabla \modelk(\iteratek), \sj \rangle| =  | \min_{\iteratek+d \in \feasiblek, \|d\| \le \|\sj\|}\langle \gradmodelk(\iteratek), d\rangle| = \chi(\iteratek, \|\sj\|),
\end{align}
so that
\begin{align}
\label{12_2_8}
\modelk(\iteratek) - \modelk(\xj) \ge \kappa_{ubs}\chi(\iteratek, \|\sj\|).
\end{align}

Next, consider the case where $\|\sj\| \ge 1$.
Then because $\chi(x, \theta)$ is continuous and nondecreasing as a function of $\theta$ for all $\theta \ge 0$ (Theorem \ref{chi_non_inc}),
we also know that
\[
\modelk(\iteratek)-\modelk(\xj) \ge \kappa_{ubs}\chik(\iteratek, 1) = \kappa_{ubs}\chik
\]
We may now assume that $\|\sj\| < 1$.
Because of \ref{12_2_8} and Theorem \ref{chi_non_inc}, 
\[
\modelk(\iteratek) - \modelk(\xj) \ge \kappa_{ubs}\chik\|\sj\|.
\]

If $\|\sj\| \ge \kappa_{frd}\Delta_k$ (condition \ref{too_small_1}), then
\begin{align}
\modelk(\iteratek) - \modelk(\xj) \ge \kappa_{ubs}\kappa_{frd}\chik \Delta_k
\end{align}

On the other hand, if \ref{too_small_2} holds, then from Taylor's theorem, there is some $\xi_k$ in the segment $[\iteratek, \xj]$ such that

\[
\modelk(\xj) = \modelk(\iteratek)  + \langle \gradmodelk, \sj \rangle + \frac 1 2 \langle \sj, \nabla^2\modelk(\xi_k)\sj\rangle
\]
This implies
\begin{align}
\langle \sj, \nabla^2 \modelk(\chik)\sj \rangle = 2 [\modelk(\xj) - \modelk(\iteratek) + |\langle \gradmodelk, \sj\rangle|] \ge 0
\end{align}

Therefore, we see that

\[
\beta_k \ge \frac {\langle \sj,\nabla^2\modelk(\chi_k)\sj\rangle}{\|\sj\|^2}
\]

\[ \ge \frac {2(1-\kappa_{lbs})|\langle\gradmodelk,\sj\rangle|}{\|\sj\|^2} \]
and once again because $\chik(\iteratek, \|\sj\|) \ge \chik \|\sj\|$

\[
\ge \frac {2(1-\kappa_{lbs})\chik}{\|\sj\|^2}
\]

This produces
\[
\modelk(\iteratek)-\modelk(\xj) \ge 2 \kappa_{ubs}(1-\kappa_{lbs})\frac{\chik^2}{\beta_k}.
\]

If \ref{too_small_3} holds but $\|\sj\|<\kappa_{frd}\Delta_k$, we know because of \ref{12_1_5_3} that
\[
\chik(\iteratek, \min[\kappa_{frd}\Delta_k, 1]) \le \|\gradmodelk, \sj\| + 2 \min[\kappa_{frd}\Delta_k, 1]\|P_{T(\xj)}[-\gradmodelk]\|\le 2 \|\langle\gradmodelk, \sj\rangle\|
\]

Also,
\begin{align}
\|\gradmodelk, \sj\| \ge \frac 1 2 \chik(\iteratek, \min[\kappa_{frd}\Delta_k,1]) \ge \frac 1 2 \kappa_{frd}\chik \min[\Delta_k, 1]
\end{align}

so that
\begin{align}
\modelk(\iteratek)-\modelk(\xj) \ge \frac 1 2 \kappa_{ubs}\kappa_{frd}\chik \min[\Delta_k, 1].
\end{align}

Finally, we must only set
\[
\kappa_{dep} = \min[\frac 1 2 \kappa_{ubs}\kappa_{frd}, 2\kappa_{ubs}(1-\kappa_{lbs})] < 1
\]



\end{proof}





\section{Algorithm}

\begin{algorithmic}
\State $t_{\text{min}} \gets 0$
\State $t_{\text{max}} \gets \infty$
\State $t_{0} \gets \frac{\Delta_k}{\|\nabla \modelk(\iteratek)\|}$
\State $j=0$
\While{true}
    \State{$p(t_j, \iteratek) \gets Proj_{\feasiblek}(\iteratek-t_j\nabla \modelk(\iteratek))$}
    \State{$\trialk(t_j) \gets p(t_j, \iteratek) - \iteratek$}
    \State Evaluate $\modelk(p(t_j, \iteratek))$
    \If{\ref{too_big_1} or \ref{too_big_2} is violated}
        \State{$t_{\text{max}} \gets t_j$}
    \ElsIf{\ref{too_small_1} and \ref{too_small_2} and \ref{too_small_3} are violated}
        \State{$t_{\text{min}} \gets t_j$}
    \Else
        \State \textbf{return}  $p(t_j, \iteratek)$
    \EndIf
    \If{$t_{\text{max}} = \infty$}
        \State{$t_{j+1} \gets 2\tj$}
    \Else
        \State{$t_{j+1} \gets \frac 1 2 (t_{\text{min}} + t_{\text{max}})$}
    \EndIf
    \State{$j \gets j+1$}
\EndWhile
\end{algorithmic}



\begin{theorem}
The algorithm terminates in a finite number of steps.
\end{theorem}

\begin{proof}
If $t_{\text{max}} = \infty \forall j$ then $\trialk(\tj) \le  \Delta_k \forall j$.
This implies that

\[\lim_{t\to\infty}\|P_{\mathcal T(\ptx)}[-\nabla \modelk(\iteratek)]\| = 0\]
by \ref{12_1_3}.
But the right hand side of \ref{too_small_3} begins positive with $j=0$ and is only increasing by \ref{12_1_5_1}.
We also know \ref{too_big_2} because if it ever did not hold, $t_{\text{max}} < \infty$.

Now suppose that $t_{\text{max}} < \infty$.
Because $\modelk$ is twice continuously differentiable, we know that there is an interval satisfying both \ref{too_big_2} and \ref{too_small_2}.
Consider the case that no points in this interval satisfy \ref{too_big_1}.
This means that \ref{too_small_2} is violated for arbitrarily small $t$, including all $t$ such that $\|\ptx-\iteratek\| \le \Delta_k$.
Then \ref{too_big_2} also holds for all such $t$.
Also, the interval of $t$ such that $\kappa_{frd}\Delta_k \le \|\ptx - \iteratek\|\le \Delta_k$ is nonempty by continuity of $\|\ptx - \iteratek\|$.
By the algorithm construction, it is easy to see that $[t_{\text{min}}, t_{\text{max}}]$ always contains this interval, while $\lim_{t\to\infty}(t_{\text{max}} - t_{\text{min}}) = 0$.

\end{proof}

\appendix

\section{Assumptions}

\paragraph{AF.1}
\begin{align}
f: \mathbb R^n \to \mathbb R
\end{align}
is twice-continuously differentiable on $\mathbb R^n$.

\paragraph{AC.1}
Each
\begin{align}
c_i(x): \mathbb R^n \to R
\end{align}
is twice-continuously differentiable on $\mathbb R^n$

\paragraph{AC.2}
\begin{align}
X  \subseteq \mathbb R^n
\end{align}
is nonempty, closed, and convex.


\paragraph{AC.7}
\begin{align}
X = \cap_{i=1}^{m} \{x\in\mathbb R^n|c_i(x) \ge 0\}
\end{align}


\paragraph{AO.1}
Regularity assumptions hold, for example the linear independence constraint qualification (LICQ).

\paragraph{AM.1}
For all $k$, $\modelk$ is twice differentiable. (We choose $\modelk$ to be quadratic.)

%\paragraph{AM.2}
%$\modelk(\iteratek) = f(\iteratek)$ for all $k$


\end{document}



